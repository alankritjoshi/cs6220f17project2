% Use the following line _only_ if you're still using LaTeX 2.09.
%\documentstyle[icml2015,epsf,natbib]{article}
% If you rely on Latex2e packages, like most moden people use this:
\documentclass{article}

% use Times
\usepackage{times}
% For figures
\usepackage{graphicx} % more modern
%\usepackage{epsfig} % less modern
\usepackage{subfigure} 

% For citations
\usepackage{natbib}

% For algorithms
\usepackage{algorithm}
\usepackage{algorithmic}

% As of 2011, we use the hyperref package to produce hyperlinks in the
% resulting PDF.  If this breaks your system, please commend out the
% following usepackage line and replace \usepackage{icml2015} with
% \usepackage[nohyperref]{icml2015} above.
\usepackage{hyperref}

% Packages hyperref and algorithmic misbehave sometimes.  We can fix
% this with the following command.
\newcommand{\theHalgorithm}{\arabic{algorithm}}

% Employ the following version of the ``usepackage'' statement for
% submitting the draft version of the paper for review.  This will set
% the note in the first column to ``Under review.  Do not distribute.''
%\usepackage{icml2015} 

% Employ this version of the ``usepackage'' statement after the paper has
% been accepted, when creating the final version.  This will set the
% note in the first column to ``Proceedings of the...''
\usepackage[accepted]{icml2015}

% The \icmltitle you define below is probably too long as a header.
% Therefore, a short form for the running title is supplied here:
\icmltitlerunning{Group 2 Project Proposal}

\begin{document} 

\twocolumn[
\icmltitle{Group 2 Project Proposal: United States Obesity Risk Factor Exploration}

% It is OKAY to include author information, even for blind
% submissions: the style file will automatically remove it for you
% unless you've provided the [accepted] option to the icml2015
% package.
\icmlauthor{Brian Desnoyers}{bdesnoy@ccs.neu.edu}
%\icmladdress{Your Fantastic Institute,
%            314159 Pi St., Palo Alto, CA 94306 USA}
\icmlauthor{Alankrit Joshi}{alankrit93@ccs.neu.edu}
\icmlauthor{Rahul Kondakrindi}{rahulkondakrindi@ccs.neu.edu}
\icmlauthor{Prasanna Vikash Peddinti}{vikash4281@ccs.neu.edu}
\icmlauthor{Junyi Wang}{wang4615@ccs.neu.edu}

% You may provide any keywords that you 
% find helpful for describing your paper; these are used to populate 
% the "keywords" metadata in the PDF but will not be shown in the document
%\icmlkeywords{boring formatting information, machine learning, ICML}

\vskip 0.3in
]

% \begin{abstract} 
% \end{abstract} 

\section{Introduction}
\label{introduction}

Obesity is a major challenge facing the healthcare system in the United States. With obesity rates rising significantly over the last thirty years and if the rise continues, the United States healthcare system is projected to pay \$150 billion annually \cite{hurt2010obesity}.

In addition to genetic factors, this increase in obesity rates, inactivity, and malnutrition has been linked to the environment. 
Environmental changes; such as car transportation, inactive jobs, carry out food, food advertisements, and food portions; can be tied to specific regions and have a significant impact on health \cite{whyobesityhealthproblem, understandingadultoverweight}. 

The main goal of this project is to explore the distribution of these risk factors across the United States and visualize how these groupings correspond to obesity and health.

\section{Dataset}
\label{dataset}

For this project, we will analyze the Nutrition, Physical Activity, and Obesity dataset provided by the Centers for Disease Control and Prevention (CDC) provides information on the percentage of the population suffering from adult obesity, as well as associated behaviors, such as poor nutrition and physical inactivity, for the nation, states, and selected sub-state districts. These data also include potential risk factor features, such as age, education, sex, and income \cite{nutphysactobesitydata}.
In addition, this dataset provides similar population information for child obesity, infant breastfeeding, active transportation, and community policy supports.

\section{Purpose}
\label{purpose}
% Questions you intend to answer go here!
The primary purpose is to correlate risk factors such as poor nutrition and physical inactivity with obesity and find if they affect population's health.

We will start by analyzing the distribution of obesity, physical activity and nutrition conditions across the United States using state-by-state data from 2011 to 2016. This will eventually lead to generating visualizations which we shall compare with those provided by the CDC. There will also be a need to link the regions which are closely associated. We plan to start analysis first on obesity statistics and then physical activity and nutrition.

The next step is to figure out how the regions have similar obesity risk factors, including physical inactivity and nutrition. This will involve comparing the risk factor regions with obesity statistics. Different socio-economic, racial and ethnic groups will also be studied to learn more how inequalities contribute to higher obesity rates in certain communities.

The final part will be to map out the prevalence and trend of obesity along with its dependence on risk factors. We aim to create multiple visualizations to better understand how such risk factors contribute to health.

All of these steps will eventually help us answer whether environmental risk factors have a definite impact on population's health and obesity.

\section{Methodology}
\label{methodology}
% Clustering of obesity stats
We will attempt to use the Nutrition, Physical Activity, and Obesity dataset to generate heatmaps of obesity, physical inactivity, and poor nutrition which can be compared to those provided by the CDC \cite{adultobesitymaps}. 

To begin to understand the best grouping of regions based on their obesity statistics, we will perform a cluster analysis. We will start this analysis by performing hierarchical clustering with a Euclidean distance metric so that we will be able to visualize and present the links between different regions through a dendrogram. 

Starting with average linkage clustering, we will also experiment with the results of complete and single-linkage clustering, explaining the reasoning for any differences. 
We will compare these results with those gained from the density-based spatial clustering of applications with noise (DBSCAN) algorithm, which has the benefits of finding arbitrarily-shaped clusters and being more robust to noise. 
We then plan to perform a similar clustering analysis for physical inactivity and malnutrition statistics.
\\\\
% Clustering by risk factor
We will then attempt to perform a clustering analysis to begin to understand ways in which geographical regions have similar obesity risk factor conditions. 
Initially, we will utilize a k-means clustering analysis for this. To do this, we will first perform a $k$-means clustering analysis for the obesity statistics, similarly to when we performed hierarchical clustering and DBSCAN. 
We will perform this analysis for several values of $k$, plotting the objective function value for each value of $k$. We will select the value for $k$ corresponding to the “kink” in the objective function and use this $k$ value for our clustering of risk factor conditions. This will allow us to compare the risk factor condition regions with those of obesity statistics. We will then experiment with clustering of risk factor conditions for other values of $k$ and plot the results.
\\\\
% Dim. red.
To better understand how much these environmental risk factor values contribute to population obesity, we then plan to perform dimensionality reduction on these features. We will start with principal component analysis (PCA) and use the first two principal components to make a two-dimensional plot of the risk factors 
We will also make three-dimensional visualizations using the first three principal components. These plots will be color-coded based on obesity prevalence groupings identified through previous analyses.

We then plan to explore similar dimensionality reduction analysis to produce the best human-readable visualization of these environmental risk factors. From this visualization, using \verb|D3.js|, we will then create a 3D interactive visualization that allows for the visualization of obesity prevalence in the United States and its dependence on risk factors. This visualization will hopefully demonstrate that environmental risk factors have a clear effect on population health.

\section{Distribution of Work}
\label{distribution}

Brian Desnoyers - Data cleaning, heatmap, hierarchical cluster analysis on obesity

Alankrit Joshi - Hierarchical analysis on nutrition and physical activity, cluster visualizations

Rahul Kondakrindi - Average linkage clustering, spatial clustering

Prasanna Vikash Peddinti - k-means clustering, clustering visualizations

Junyi Wang - Dimensionality reduction analysis, D3.js visualizations

% In the unusual situation where you want a paper to appear in the
% references without citing it in the main text, use \nocite
%\nocite{langley00}

\bibliography{proposal}
\bibliographystyle{icml2015}

\end{document} 


% This document was modified from the file originally made available by
% Pat Langley and Andrea Danyluk for ICML-2K. This version was
% created by Lise Getoor and Tobias Scheffer, it was slightly modified  
% from the 2010 version by Thorsten Joachims & Johannes Fuernkranz, 
% slightly modified from the 2009 version by Kiri Wagstaff and 
% Sam Roweis's 2008 version, which is slightly modified from 
% Prasad Tadepalli's 2007 version which is a lightly 
% changed version of the previous year's version by Andrew Moore, 
% which was in turn edited from those of Kristian Kersting and 
% Codrina Lauth. Alex Smola contributed to the algorithmic style files.  
